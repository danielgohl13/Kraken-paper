\label{chap:trab_cor}

\section{DeepFish: Accurate underwater live fish recognition with deep architecture}

\section{Automatic underwater image pre-processing}

\section{Foreground Extraction of underwater Videos via Sparse and Low-rank Matrix Decomposition}

\section{A Vision Based System for Object Detection in Underwater Images.}
\subsection{resumo}
O artigo propõe um sistema (véiculo autonomo) para detecção e rastreio de objetos submersos em água, o sistema faz detecção automática de canos (que podem se extender por kilometros). Utilizando procedimento para compensasão de cor foi desenvolvido para diminuir problemas causados pela luz, a identificação dos objetos é feita por uma rede neural, que classifica os pixels em tempo real, essa rede neural ajuda a navegação automática do veículo, traçando rotas e caminhos atráves das imagens processadas. Ressonância  geométrica é usada para diminuir a quantidade de detecções falsas, melhorar a precisão e mapear o ambiente, os sistemas de sonar foram desenvolvidos para melhorar a precisão da localização do veículo, não eliminando a utilização de um trasnponder.

\subsection{conclusão}
O sistema consegue detectar canos e outras estruturas, mesmo com os problemas causados pela falta de luminosidade, dispersão e atenuação da luz e outros problemas como areia sobre canos ou outros detritos; 